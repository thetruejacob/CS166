\documentclass{article}
\usepackage[left=2cm, right=2cm, top=0cm]{geometry}
\usepackage{amsmath}
\usepackage{amssymb}
\usepackage{graphicx}
\usepackage{tikz}
\usepackage{enumitem}
\usepackage{hyperref}
\setlength\parindent{0pt}
\hypersetup{
    colorlinks,
    citecolor=green,
    filecolor=black,
    linkcolor=blue,
    urlcolor=blue
}

\begin{document}
\title{Assignment 2}
\author{Jacob Puthipiroj}
% \date{}
\maketitle

\section*{Part 1: Traffic Jams on a circular road}
The code to this section is available in the appendix

We are citing \cite{nagel1992cellular}.

Over time, we see that traffic jams can appear with sufficient traffic density. However, depending on the rule of the model, these can eventually disappear.

When using the 'always in the middle' rule, we see that with up to 50\% traffic density, jams eventually disappear. 

Try to look for the relationship between pslow, density, and max speed. 

Following the Nagel-Schrenkberg model (1992), we see... 

\section*{Part 2: Multi-Lane Highways}

In this section, we implement Rickert's (1996) extension to the Nagel-Schrankenberg model. Compared to a single lane road, 

Visualizations over time. 

Analyze how much more traffic can flow through a multilane compared to a single lane model (Plot), latexify


How applicable is the model to Buenos Aires?
- Buenos Aires is a city with wide, open streets. 

How can bad driver behavior be encoded in our algorithm?
- Define bad driver behavior

Extend the model to include speed limits, traffic lights, obstacles, and lane merges. 

\section*{Future Work}













\newgeometry{left =2cm, right = 2cm}



\bibliographystyle{apacitex}
\bibliography{bibfile}

\section*{Appendix}
\subsection*{Single Lane Traffic Flow}
The code to this section is available here

\subsection*{MultiLane Traffic Flow}




\end{document}